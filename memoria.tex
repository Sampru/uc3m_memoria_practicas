\documentclass[titlepage]{article}
\usepackage[utf8]{inputenc}
\usepackage[spanish]{babel}

% Header
\usepackage{fancyhdr}
\pagestyle{fancy}
\fancyhf{}
\fancyhead[L]{Asier Sampietro Alberdi}
\fancyhead[R]{\leftmark}
\fancyfoot[R]{\thepage}

% Graficos
\usepackage{graphicx}
\graphicspath{ {images/} }

% Secciones
\makeatletter
\newcommand\level[1]{%
  \ifcase#1\relax\expandafter\chapter\or
    \expandafter\section\or
    \expandafter\subsection\or
    \expandafter\subsubsection\else
    \def\next{\@level{#1}}\expandafter\next
  \fi}
\newcommand{\@level}[1]{%
  \@startsection{level#1}
    {#1}
    {\z@}%
    {-3.25ex\@plus -1ex \@minus -.2ex}%
    {1.5ex \@plus .2ex}%
    {\normalfont\normalsize\bfseries}}

\newdimen\@leveldim
\newdimen\@dotsdim
{\normalfont\normalsize
 \sbox\z@{0}\global\@leveldim=\wd\z@
 \sbox\z@{.}\global\@dotsdim=\wd\z@
}

\newcounter{level4}[subsubsection]
\@namedef{thelevel4}{\thesubsubsection.\arabic{level4}}
\@namedef{level4mark}#1{}
\def\l@section{\@dottedtocline{1}{0pt}{\dimexpr\@leveldim*4+\@dotsdim*1+6pt\relax}}
\def\l@subsection{\@dottedtocline{2}{0pt}{\dimexpr\@leveldim*5+\@dotsdim*2+6pt\relax}}
\def\l@subsubsection{\@dottedtocline{3}{0pt}{\dimexpr\@leveldim*6+\@dotsdim*3+6pt\relax}}
\@namedef{l@level4}{\@dottedtocline{4}{0pt}{\dimexpr\@leveldim*7+\@dotsdim*4+6pt\relax}}

\count@=4
\def\@ncp#1{\number\numexpr\count@+#1\relax}
\loop\ifnum\count@<100
  \begingroup\edef\x{\endgroup
    \noexpand\newcounter{level\@ncp{1}}[level\number\count@]
    \noexpand\@namedef{thelevel\@ncp{1}}{%
      \noexpand\@nameuse{thelevel\@ncp{0}}.\noexpand\arabic{level\@ncp{1}}}
    \noexpand\@namedef{level\@ncp{1}mark}####1{}%
    \noexpand\@namedef{l@level\@ncp{1}}%
      {\noexpand\@dottedtocline{\@ncp{1}}{0pt}{\the\dimexpr\@leveldim*\@ncp{5}+\@dotsdim*\@ncp{0}\relax}}}%
  \x
  \advance\count@\@ne
\repeat
\makeatother
\setcounter{secnumdepth}{100}
\setcounter{tocdepth}{100}
% Secciones end

\setlength{\parskip}{1em}

\title{
   Almis Informática Financiera S.L.\\
    \textbf{Memoria de prácticas}
}
\author{
    Asier Sampietro Alberdi\\
    Tutor: José Manuel Nuñez
}
\date{28 de marzo, 2019}

\begin{document}
\begin{titlepage}
\maketitle
\end{titlepage}

\section*{Resumen del trabajo}
El trabajo realizado en Almis Informática Financiera S.L. (referido como Almis en adelante) ha tratado sobre la ampliación de las funcionalidades de una aplicación financiera. En concreto, se ha trabajado en el apartado de tratamiento de eventos sobre renta variable que dispone la ampliación. En el documento se detallarán los siguientes apartados: definición del problema, las actividades realizadas, las conclusiones y las competencias trabajadas.
\newpage

\tableofcontents
\newpage
\listoffigures
\newpage

\level{1}{Definición del problema}
La aplicación principal de Almis (en adelante referido como FIT, Financial Intelligence Tool) es un gestor de carteras usado mayormente para gestionar carteras en agencias de valores. Esta aplicación cuenta con numerosas herramientas usadas día a día por gestores, pero es una herramienta que sigue en desarrollo. Las principales necesidades de FIT son labores de desarrollo y de mantenimiento.\par

El apartado del desarrollo esta enfocado de cara a cliente. Este manda peticiones y se desarrollan soluciones a medida o globales, que podrían interesar a otros usuarios. Aunque las peticiones tienden a ser variadas y se opte por hacerlas a medida, el último año se han centrado en funcionalidades sobre los eventos de renta tanto fija como variable.\par

El apartado de mantenimiento, en cambio, se centra en el departamento de QoS\footnote{\emph{Quality of Service}. Del inglés, ofrecer un servicio de calidad.}. FIT es una herramienta que ha crecido mucho, y a medida que se añaden procesos, el cómputo general se ralentiza cada vez más. Esto provoca una mala experiencia de usuario, y en FIT se trabaja para que esto suceda lo menos a menudo posible.

\level{2}{Desarrollos}
La demanda de este año ha ido enfocado a un robusto módulo de gestión de eventos. Los clientes quieren automatizar cada vez más eventos relacionados principalmente con la renta variable. Debido al aumento de inversores de pequeña escala gracias a las aplicaciones de trading, los clientes se enfrentan a una demanda mayor de eventos sobre las acciones de los clientes.\par

Estos desarrollos, al ser principalmente cálculos realizados cuando las carteras pasan de día\footnote{Cuando una cartera pasa de día se calculan los resultados de las operaciones que vencen a dicha fecha, y se actualiza su posición en base a ello.}, se componen de tareas relacionadas con el back-end\footnote{Parte de la aplicación web que no visible, encargado de cálculos y gestión de los datos.}, siendo el front-end\footnote{Parte de la aplicación web que visualiza el usuario, con una carga lógica mínima.} casi idéntico. Por este motivo, plantea añadir los nuevos eventos a una ventana web ya en uso para eventos similares.\newpage

\level{3}{Eventos}
Tras reuniones con clientes, se ha optado por priorizar 3 eventos, uno de los cuales necesita ser mejorado y otros dos que deben ser desarrollados desde cero.

\level{4}{Derechos de suscripción}
Este evento es el que esta desarrollado parcialmente. La implementación se hizo para salir del paso en su momento, y ahora se quiere mejorar porque el uso ha aumentado.\par

Este evento sucede cuando una empresa decide hacer una ampliación, para evitar que el porcentaje de participación de un accionista caiga o diluya. Para ello, los accionistas cambian sus posiciones a derechos. De esta manera, tienen una posición temporal mientras se amplia el capital. Al terminar, los derechos se transforman de nuevo en acciones, en este caso en las nuevas acciones, con el porcentaje que tendría el accionista en el estado actual. Los derechos de suscripción son instrumentos comerciables. Se pueden vender en el mercado como si fueran acciones; recibir dividendos; o venderlos al emisor, que es en lo que se diferencian principalmente a las acciones.\par

Y es este caso de uso el que necesita adoptar FIT. Actualmente el emisor de los derechos no puede establecer un precio al cual se venderán las acciones de los interesados en una fecha dada, por lo que se requiere un desarrollo sobre el código que ya está en producción. Aprovechando esto, también se retocara la ventana, ya que como se ha mencionado anteriormente, se hizo con poco detalle para salir de paso.\par


\level{4}{Primas de asistencia}
También relacionado con los accionistas, el segundo evento a tratar son las primas a los asistentes de las juntas de accionistas. El emisor de este evento definirá un precio que pagara por cada acción que posea el asistente.\par

Al tratarse de un evento simple (a nivel de interfaz), se piensa incluir en la ventana de pagos y cobros. Esta ventana es la encargada de gestionar diferentes fluctuaciones en las posiciones de las carteras gestionadas. Cuando se crea un evento como este, se valida o rechaza desde esta ventana, creando así el flujo de caja correspondiente.\newpage

\begin{figure}[h]
\centering
\includegraphics[width=\textwidth]{cobros_pagos}
\caption{Ventana de pagos y cobros}
\end{figure}

Como se puede observar en la figura 1, la ventana cuenta con un selector de eventos (o cobro o pago, como se le llama en la ventana), y puedes seleccionar varios para confirmarlos de una tacada. Desde el botón de 'Detalle' se pueden hacer ligeras modificaciones previas a la confirmación. También existe la posibilidad de descartarlos, rechazando la propuesta de la aplicación y teniendo que meterlo de forma manual posteriormente.

\level{4}{Prima de emisión}
Relacionado con los derechos de suscripción, los clientes también solicitan la posibilidad de ejercer una prima de emisión a la hora de darse una ampliación de capital.\par

La prima de emisión es la parte de la aportación que hay que realizar para suscribir una acción o participación que viene dada por diferencia entre su valor de emisión y su valor nominal. En otras palabras: la prima de emisión es la cantidad que hay que pagar para adquirir una acción o participación en el momento de su emisión además de su valor nominal. De esta manera, los accionistas más antiguos conservarán su ventaja manteniendo todos una parte equitativa de la empresa a lo aportado.\newpage

\level{2}{Tareas de mantenimiento}
FIT es una aplicación que no para de desarrollarse junto con el motor que le brinda un entorno web. La evolución de continua de estos dos hace que los apartados más viejos se vean realmente perjudicados, y por eso se deben hacer varias tareas de mantenimiento de manera constante para cumplir con la promesa de calidad con el cliente. Por ello se han marcado varias ventanas como candidatas a una refactorización.\par

Como se ha comentado anteriormente, el foco de este año ha ido fijado en los eventos sobre posiciones, y es por esto que se ha decidido mejorar varias ventanas relacionadas con estos.\newpage

\level{3}{Ventana de pagos y cobros}
La ventana anteriormente mencionada, cuenta con diversos eventos que requieren tratos específicos. Como se puede observar en la figura 2, la ventana cuenta con demasiados eventos como para dar un trato específico para cada uno de ellos, por lo que es necesario un trabajo de homogeneización.

\begin{figure}[h]
\centering
\includegraphics[scale=0.6]{tipo_cobros_pagos}
\caption{Eventos disponibles en la ventana de pagos y cobros.}
\end{figure}

Esta labor no cambiará nada a nivel visual, pero se encargará de brindar una navegación mas agradable para el usuario.\newpage

\level{3}{Ventanas de definición de eventos de renta variable}
La filosofía de Almis respecto al reciclaje de código ha sido el causante de esta necesidad. Como todos los eventos se declaran de manera similar. Siguiendo el afán de reutilización, se optó por crear una ventana genérica para todos los eventos, y gestionar mediante dependencias de AngularJS\footnote{Un motor desarrollado en JavaScript que permite dinamizar las ventanas web.} las diferencias que podían darse, añadiendo coste computacional al navegador.\par

Llegados a cierto punto, esta estrategia ha resultado inviable, puesto que ciertos clientes (debido a la multitud de datos que cotejan) son incapaces de gestionar nada mediante estas ventanas, ya que el navegador no soporta tal carga.\par

A son de este problema, se ha optado por dividir la ventana base en múltiples, tratando cada evento desde la ventana misma y reduciendo las inyecciones de AngularJS.\par

\begin{figure}[h]
\centering
\includegraphics[scale=0.3]{ventanas_optimizadas}
\caption{Menú con las ventanas optimizadas.}
\end{figure}

Como se puede observar en la figura 4, todas estas ventanas parten de la misma base. Se duplicará el código y se modificará cada caso para habilitar estas ventanas al uso de nuevo.

%---------------------------------------------------------

\level{1}{Actividades realizadas}
Para enfrentarse al problema descrito, se han realizado las siguientes tareas en los ámbitos de desarrollo y mantenimiento de la aplicación.

\level{2}{Desarrollo de eventos}
Como se ha mencionado anteriormente, dentro de todos los eventos que recoge o plantea recoger FIT, se ha puesto el foco en tres. Estos son los desarrollos que se han realizado para cada uno para ofrecer una solución a las necesidades de los clientes.\par

\level{3}{Primas de asistencia}
La gestión de las primas de asistencia ha sido desarrollada desde cero. Como se ha mencionado en el planteamiento del problema, la ventana de pagos y cobros ha sido la elegida de albergar esta funcionalidad, así que lo primero que se ha hecho es compatibilizar la tabla de la base de datos que refleja esta ventana.\par

Los datos se separan en dos tablas: una que recoge los datos generales del evento, como fecha de vencimiento, importe de la prima o datos del emisor; y otra que guarda temporalmente los detalles del evento y se calcula y rellena en tiempo de ejecución.\par

\begin{figure}[h]
\centering
\includegraphics[width=\textwidth]{primas_asistencia}
\caption{Listado de primas de asistencia.}
\end{figure}

Tras validar que ambas tablas son aptas para el desarrollo, se ha pasado a la parte de visualización. Como se ha mencionado, la visualización conlleva una serie de cálculos que ofrecen detalles sobre el evento, como retenciones y comisiones, junto con el efectivo neto. Se ve más claro en el ejemplo de las figuras 4 y 5.\par

\begin{figure}[h]
\centering
\includegraphics[width=\textwidth]{detalle_primas_asistencia}
\caption{Detalles de las primas de asistencia.}
\end{figure}

Una vez conseguido visualizar los datos, se pasa a la ejecución o confirmación del evento. Esto se realiza mediante una función de C donde se calculan los flujos que va a generar esta operación, y se dejan almacenados en una tabla de la base de datos. Esto se hace así porque los resultados se calculan al final del día, junto con el resto de cálculos.\par

\level{3}{Primas de emisión}
Las primas de

\level{3}{Derechos de suscripción}
\level{2}{Optimización de ventanas}



\end{document}
